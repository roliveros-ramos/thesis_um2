Este trabajo representa una contribución original a la metodología para el desarrollo de modelos ecosistémicos así como el primer intento de desarrollar un modelo de extremo a extremo (E2E, \emph{end-to-end}) para el Norte del Ecosistema de la Corriente de Humboldt (NECH). El principal propósito del modelo desarrollado es construir una herramienta para el manejo ecosistémico y la toma de decisiones, razón por la cual la credibilidad del modelo es esencial, y ésta puede ser evaluada mediante la confrontación con datos. Adicionalmente, el NECH muestra una alta variabilidad climática y oceanográfica a diversas escalas, siendo la mayor fuente de variabilidad interanual la interrupción de la estacionalidad del afloramiento por El Niño-Oscilación Sur, que tiene efectos directos en la sobrevivencia larval y el éxito del reclutamiento de peces. La actividad pesquera también puede ser altamente variable, dependiendo de la abundancia y accesibilidad de los principales recursos pesqueros. Este contexto genera las dos principales preguntas metodológicas abordadas en esta tesis, a través del desarrollo de un modelo de extremo a extremo por el acoplamiento del modelo de nivel trófico alto OSMOSE y el modelo hidrodinámico y biogeoquímico ROMS-PISCES: i) cómo calibrar modelos ecosistémicos usando series de tiempo y ii) como incorporar el impacto de la variabilidad interanual del ambiente y la pesca.

Primero, esta tesis resalta algunos problemas relacionados a la confrontación de modelos ecosistémicos complejos con datos, y propone una metodología para la calibración secuencial de modelos ecosistémicos. Proponemos dos criterios para la clasificación de parámetros de un modelo: la dependencia al modelo y la variabilidad temporal de los parámetros. Luego, estos criterio en conjunto con la disponibilidad de valores iniciales aproximados para los parámetros son usados como reglas de decisión para determinar qué parámetros necesitan ser estimados y su orden de precedencia en el proceso de calibración secuencial. Adicionalmente, un nuevo algoritmo evolutivo diseñado para la calibración de modelos estocásticos (e.g. Modelos Basados en Individuos) y optimizado para la estimación por máxima verosimilitud ha sido desarrollado y aplicado a la calibración del modelo OSMOSE con datos de series temporales. 

La variabilidad ambiental es explícita en el modelo: ROMS-PISCES forza al modelo OSMOSE y dirige los potencial efectos \emph{bottom-up} a la red trófica a través de las interacciones entre el plankton y los peces, así como a través de los cambios en la distribución espacial de los peces. Esto último fue tomado en cuenta usando modelos de distribución de especies que son tradiacionalmente evaluados a través de una matriz de confusión y las métricas estadísticas asociadas a esta. Sin embargo, cuando se considera la predicción del habitat en el tiempo, la variabilidad espacial de la distribución espacial puede ser resumida y validada usando los patrones emergentes de la forma de la distribución espacial. Nosotros modeladmos el habitat potencial de las principales especies del NECH usando varias fuentes de información (pesquerías, cruceros científicos y seguimiento satelital de los barcos) conjuntamente con datos ambientales de sensoramiento remoto y observaciones \emph{in situ}, desde 1992 a 2008. El habitat potencial fue predicho con resolución mensual, y el modelo fue validado usando información cuantitativa y cualitativa del sistema usando un enfoque orientado a patrones. 

El modelo de extremo a extremo ROMS-PISCES-OSMOSE para el NECH fue calibrado usando nuestro algoritmo evolutivo y un enfoque de máxima verosimilitud para ajustar datos de series de tiempo de desembarques, índices de abundancia y capturas por longitud y edad de 1992 a 2008. Para concluir, algunas aplicaciones potenciales del modelo al manejo pesquero son presentadas y sus limitaciones y perspectivas discutidas. 

