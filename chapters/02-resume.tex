Ce travail représente une contribution originale à la méthodologie pour le dé\-ve\-lo\-ppe\-ment de modèles écosystémiques ainsi qu'une première tentative d'une modélisation end-to-end (E2E) del'écosystème du Courant de Humboldt Nord (NHCE: Northern Humboldt Current Ecosystem). L'objectif principal du modèle développé dans cette thèse est de construire un outil de gestion écosystémique et d'aide à la décision ; raison pour laquelle la crédibilité du modèle est essentielle, laquelle peut-être établie par confrontation aux données. En outre, le NHCE présente une grande variabilité climatique et océanographique à différentes échelles, la source principale de variation inter-annuelle étant l'interruption du cycle d'upwelling saisonnier par l'Oscillation Australe du phénomène El Niño (ENSO: El Nino Southern Oscillation)qui a un effet direct sur la survie larvaire et le succès de recrutement des poissons. La pêche peut aussi être fortement variable, en fonction de l'abondance et de l'accessibilité des principales ressources halieutiques. Ce contexte amène deux questions méthodologiques principales que nous explorons dans cette thèse à travers le dévelo\-ppe\-ment d'un modèle E2E qui couple le modèle OSMOSE, pour la partie haut niveau trophique, au modèle ROMS-PISCES, pour les parties hydrodynamique et biogéochimie:(i) Comment calibrer un modèle écosystémique à partir de séries temporelles de données ? (ii) Comment inclure l'impact de la variabilité inter-annuelle de l'environnement et de la pêche ?


En premier lieu, cette thèse met en évidence plusieurs problèmes liés à la confrontation de modèles écosystémiques complexes aux données et propose une mé\-tho\-do\-lo\-gie pour une calibration séquentielle en plusieurs phases des modèles éco\-sys\-té\-mi\-ques. Nous proposons deux critères pour classer les paramètres d'un modèle: la dépendance au modèle et la variabilité temporelle des paramètres. A partir de ces critères, et en tenant compte de l'existence d'estimations initiales, on énonce des règles qui permettent de déterminer quels paramètres doivent être estimés, et dans quel ordre, dans le processus de calibration séquentiel. De plus, un nouvel Algorithme Évolutionnaire, conçu pour la calibration de modèles stochastiques (tels les modèles individu-centré) et optimisé pour l'estimation du maximum de vraisemblance, a été développé et utilisé pour la calibration du modèle OSMOSE avec des séries temporelles de données.


La variabilité environnementale est explicite dans le modèle: le modèle ROMS-PISCES force le modèle OSMOSE et propage les effets bottom-up potentiels dans le réseau trophique à travers les interactions trophiques entre plancton et poisson d'une part, et les changements dans la distribution spatiale du poisson d'autre part. Cette dynamique spatiale des poissons est prise en compte par l'utilisation de modèles de distribution des espèces de type présence/absence, qui sont en général évaluésgrâce à une matrice de confusion et les indicateurs statistiques qui lui sont associés. Toutefois, quand on considère la prédiction d'un habitat au cours du temps, la variabilité de la distribution spatiale des habitats peut être résumée de manière complémentaire et validée en utilisant les patrons émergents de la forme des distributions spatiales. Nous avons modélisé l'habitat potentiel des principales espèces du NHCE en utilisant plusieurs sources d'information (pêches commerciales, campagnes scientifiques et suivi satellite des navires de pêche) conjointement aux données environnementales issues d'observations satellites et in-situ, de 1992 à 2008. L'habitat potentiel est estimé sur cette période d'étude avec une résolution mensuelle, et le modèle est validé à partir d'informations quantitatives et qualitatives du système, en utilisant une approche pattern-oriented.


Le modèle écosystémique E2E ROMS-PISCES-OSMOSE pour le NHCE est calibré en utilisant notre algorithme évolutionnaire et une approche par maximum de vraisemblance pour ajuster des séries temporelles mensuelles de données de débarquements, d'abondances et de captures par classes de taille de 1992 à 2008. En conclusion, quelques applications potentielles du modèle pour la gestion des pêches sont présentées et nous discutons leurs limitations et les perspectives. 
