
In this chapter we address the key calibration phase of the end-to-end (E2E) ecosystem model ROMS-PISCES-OSMOSE for the Northern Humboldt Current Ecosystem. For this purpose, we highlight some issues related to the confrontation of complex ecosystem models to data and propose a methodology for a sequential multi-phases calibration of ecosystem models (section \ref{paper1}). We first propose two criteria to classify the parameters of a model: the model dependency and the time variability of the parameters. Then, these criteria and the availability of approximate initial estimates are used as decision rules to determine which parameters need to be estimated, and their precedence order in the sequential calibration process. 

The E2E model is calibrated using an the evolutionary algorithm described in chapter \ref{EA} and a likelihood approach to fit monthly time series data of landings, abundance indices and catch at length distributions from 1992 to 2008.  

\section{A sequential approach for the calibration of ecosystem models}
\label{paper1}

In this section we include a manuscript submitted to the journal ``Progress in Oceanography'' (manuscript PROOCE-S-14-00115).


\includepaper{papers/oliveros_etal-PiO-sequential_calibration_for_thesis.pdf}


